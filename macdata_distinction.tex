\documentclass[12pt]{article}
\usepackage[margin=3cm]{geometry}

% \usepackage{etex} % fix for using xy and tikz in the same document
\usepackage[utf8]{inputenc}
\usepackage{parskip}
\usepackage{enumerate}
\usepackage{hyperref}

\begin{document}
\section*{Distinction between MacDATA proposal and PhD project}

Many techniques from statistical methods (SM) and machine learning (ML) overlap. However, SM has a well established focus on inference by building probabilistic models which allows us to determine a quantitative measure of confidence about the clarity of the true effect. Simulation-based validation approaches can be used in conjunction with SM to explicitly verify assumptions and redefine the specified model, if necessary. On the other hand, ML uses general-purpose algorithms to find patterns that best predict the outcome and makes minimal assumptions about the data-generating process; and may be more effective in a number of situations. Despite convincing predictive performance and flexibility, however, the lack of explicit models in most ML methods can make ML results difficult to directly link to prior public health knowledge.

The candidate's PhD thesis and the proposed project share the idea of validated learning about public-health questions, but they emply different perspectives, and diferent computational approaches, which are applied to different questions:

\begin{itemize}
\item Aims and scope of PhD thesis
\begin{itemize}
\item Using machine learning methods to build, validate and compare prognostic ML models to predict survival in patients with cancer using both clinical and patient reported outcomes, and incorporating changes in the measured covariates over time using unique databases available in Ontario, Canada
\end{itemize}
\item Aims and scope of MacDATA project
\begin{itemize}
\item A multivariate multilevel analysis for binary outcomes: correlates of sanitary improvements in the slums of Nairobi
\end{itemize}
\end{itemize}
The first case focuses on ML-based cross-validation to evaluate predictive models for maximized predictive performance on unobserved outcomes. In the second case we propose statistical modeling together with simulation-based validation approaches to focus on causal inference of model parameters.

\end{document}
