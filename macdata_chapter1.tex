\section*{Introduction}

Many residents of urban areas face multiple obstacles to basic service access, including social-economic, institutional, spatial and political barriers, and these are even more prevalent and severe in slum settlements \citep{pierce2017basic}. In Kenya, the Nairobi Urban Health and Demographic Surveillance System (NUHDSS) has collected data since 2003 on household-level access to water, sanitation and hygiene (WaSH) services in two large slum areas \citep{beguy2015health}. The NUHDSS data provides a useful starting point for understanding the factors associated with trends in access to WaSH services in slum areas.

Our main study outcome is measured by three variables (improved water, toilet facilities and garbage collection). This is a challenging data set for a number of reasons.
We expect outcome variables to be correlated, beyond factors explained by our predictors, due to causal interactions and unmeansured covariates). We also expect temporal correlations between measurements from each household. Dealing with these considerations is complicated by the fact that the outcome variables are binary, and we thus cannot use standard techniques based on outcome normality.
We would like a modeling approach that can: 
\begin{enumerate}
\item Control for correlations within househoulds
\item Address correlation between outcomes
\item Estimate outcome-specific effects
\end{enumerate}

One way to address 1 and 2 would be combining outcome indicators into scores (representing an aggregated assessment of outcomes for each household), and then regressing the scores against the covariates. This method fails to address 3: it shows only how predictors connect to the aggregate outcome, not with particular outcomes. Univariate-response models -- with separate mixed logistic linear models being fitted for each outcome, for example -- address 3 at the expense of 1 and 2. Specifically, separate models will ignore the fact that the outcomes observed from the same subject (household) are likely to be correlated, since they are subject to shared influences that are distinctive to that particular household \citep{lalonde2019disc}. Ignoring such correlations may lead to poor estimates \citep{dassimultaneous, fang2018joint, ivanova2016mixed, miro2004bayesian, lalonde2019disc}.

Joint modeling of multiple outcomes is sometimes preferable to separate models especially in longitudinal studies \citep{dassimultaneous,fang2018joint,lalonde2019disc}. There are a number of advantages to this approach. First, in spite of its simultaneous formulation, by correctly nesting the effect of interest (for example, a particular covariate) within the multiple outcomes, both outcome-specific and global effects can be estimated \citep{dassimultaneous,lalonde2019disc}. Second, association between outcomes can be captured in terms of correlation between household-level random effects \citep{ivanova2016mixed}. Third, joint modeling may increase statistical power \citep{dassimultaneous}. But as noted above, correctly formulating and fitting these model is particularly challenging in our case of binary outcomes.

\section*{Research goals}

We will develop, validate and apply a joint modeling approach to analyse binary outcomes. Our approach will take into account the longitudinal nature of the data to model all three WaSH outcome variables (improved water, toilet facilities and garbage collection). The analysis will be based on a generalized linear mixed model approach. We will compare univariate- and multivariate-outcome models. By investigating the contribution of demographic and economic factors to WaSH access, we hope to identify overlooked factors that will be helpful to government and other development workers seeking to extend and equalize access.

\section*{Interdisciplinary nature}

Multivariate modeling of WaSH indicators is an inherently interdisciplinary question because of its intersection with several paradigms -- public health, statistics and data science. The proposed approach will be informed by public-health expertise – for example, on what factors may lead to correlations in the outcome variables beyond those explained by predictors. The candidate, supervisors, and committee will work with subject-matter experts based in Kenya both to refine the analysis approach and to interpret the results and their potential implications for improving development and for guiding future research questions. The statistical questions of validating and correctly applying these applying are at the heart of the project. Data science methods will also be important. We will carefully clean and curate the NUHDSS data, and will aim to have a complete analysis pipeline that will make it easy to replicate our analysis with updated data, or with different assumptions, should that become necessary.

\section*{Methodology}

\subsection*{Data acquisition}

We will use existing data from a longitudinal NUHDSS covering two major urban slums in Nairobi, Kenya. The baseline survey that defined the initial population for the NUHDSS was carried out from July–August 2002. Subsequently, demographic, socio-economic, household characteristics  and livelihood sources data were collected yearly until the end of 2015. \citet{beguy2015health} gives a complete description of the NUHDSS study design and setting.

\textbf{Outcomes}

We will focus on three WaSH variables: Drinking water source; Toilet facility type; and Garbage disposal method. Each of these is classified as improved or unimproved -- this follows WHO guidelines \citep{journal.pone.0151645}.

\textbf{Covariates}

Below are some of the covariates which will be included in our model:

\begin{itemize}
\item \textbf{Demographics}
\begin{itemize}
\item Age
\item Gender
\item Slum area (Korogocho and Viwandani)
\item  Number of household members (in this + other structure)
\item The interview year
\end{itemize}
\item \textbf{Socio-economic}
\begin{itemize}
\item Dwelling index
\item Assets ownership index
\item Household income
\item Total expenditure
\item Household food consumption
\item Shocks index
\item Household self rating
\end{itemize}
\end{itemize}

\subsection*{Analysis plan}

We will fit a generalized mixed model (using both classical frequentist mixed-model and Bayesian frameworks), i.e., joint hierarchical logistic regression model with shared random effects that accounts for the correlation due to households and years. Specifically, the shared random intercept will account for the correlation between the three outcomes of the same household in a particular year, and captures the unobserved factors specific to each household (in a particular year) that may influence the three services.

In order to explore and understand the two fitting approaches, we will perform simulation-based validation.  The main idea is to validate the proposed model and redefine the it, if necessary using simulated data.

\section*{Significance}

While WaSH services constitute some of the most basic requirements for human health and dignity throughout the world, this dignity is missing in most of the slum areas of Nairobi. Even though there have been a number of interventions intended to upgrade Nairobi's slum areas, focusing on issues of infrastructure development, especially in the Korogocho and Viwandani slums, the communities remain exposed to overcrowding and poverty, alcohol and substance abuse, domestic violence and crime. In addition to widespread poverty, residents of Nairobi's Viwandani and Korogocho slum areas are also faced by the near-absence of most of the basic services they need to live healthy lives. It is also important to note that Kenya has experienced unprecedented urban growth, which is expected to lead to the country's urban population reaching about 31.7 million (56\%) by 2027. This rapid urbanization has left Kenyan cities with a huge unmet demand for critical infrastructure and basic services, adversely affecting the quality of life for urban residents, with nearly two-thirds of urban residents having no access to improved sanitation \citep{chikozho2019leaving}. 

The results from this study will provide an overview to the extent to which residents of Nairobi's slum areas of Viwandani and Korogocho have been able to access and the status of WaSH services. We will work with Kenya-based subject-matter experts to frame the study; interpret the results; and plan future research based on these resutls. We are hopeful that this research can be used to inform the agenda of policymakers and public health practitioners who grapple continuously with the challenges faced in accessing WaSH services in Nairobi's low-income residential areas. It will also directly contributes to the growing body of knowledge on access to improved WaSH services in the context of slum areas in low- and middle-income countries.

